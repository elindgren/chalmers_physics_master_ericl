\chapter{Interaktion mellan student och artificiell intelligens}
\label{sec: inter}

Med hjälp av webbsidan YATA som beskrevs i kapitel \ref{sec: webb} och den utvecklade artificiella intelligensen i kapitel \ref{sec:Deep} presenteras här en kvalitativ undersökning av studenters interaktion med den framtagna artificiella intelligensen. Specifikt avser vi med begreppet interaktion studenternas reaktion och tillit till en förutsägelse av deras studieresultat. I undersökningen används osäkerhetsnätverken från avsnitt \ref{sec:uncert}, varav förutsägelserna som presenteras för studenterna är uppskattade skrivningspoäng. I kapitlet presenteras inledningsvis undersökningens bakgrund och utformning och sedan resultatet av experimentet.

\section{Bakgrund}

För att undersöka studenters tillit till förutsägelsen behöver vi precisera vad tillit innebär. Konceptet tillit är en abstrakt mental inställning och har studerats både inom socialpsykologi och informationsergonomi \cite{ergonomi}. Tillit utgör en central del av förhållanden människor emellan men även i hur vi förhåller oss till teknologi. Utvecklare av komplexa tekniska system såsom autopiloter i flygplan eller självkörande bilar har ett intresse av att mäta tillit, för att på ett framgångsrikt sätt kunna implementera systemen. Ett kvantitativt mått på tillit har därför varit nödvändig.

Det har gjorts flera studier med syfte att ta fram ett kvantitativt mått på tillit. Studierna indikerar att människors tillit, vare sig den är till en annan människa eller till ett automatiserat system, är beroende av en mängd olika faktorer till exempel trovärdighet, förutsägbarhet och uppriktighet. Flera tillitsformulär har arbetats fram för att innesluta dessa komponenter av tillit. Ett sådant formulär togs fram i arbetet \emph{Foundations for an empirically determined scale of trust in automated systems} \cite{trust}. 

Utöver tillitsaspekten utgör reaktionen på förutsägelsen en del av interaktionen. Reaktionen kan delas in i en direkt och en långsiktig komponent. Med den direkta reaktionen syftar vi på exempelvis ansiktsuttryck, fysiska reaktioner och direkta kommentarer i samband med förutsägelsen. Den långsiktiga reaktionen innefattar till exempel ändrade studiemönster som konsekvens av förutsägelsen. 

\section{Metod}
Nedan redovisas urvalsprocessen och experimentet som genomfördes för att undersöka interaktionen mellan student och artificiell intelligens. Först presenteras urvalsprocessen och därefter tillvägagångssättet för experimentet.

För undersökningen fanns tillgängliga studenter från kursen TMS064 Matematisk statistik. Precis som tidigare benämns kursen som kurs C. Urvalet av studenter som tillfrågades för att få möjligheten att ta del av respektives förutsägelse gjordes utifrån ett villkor. Kriteriet var studenternas aktivitet på webbsidan YATA, eftersom en högre aktivitet innebär större datamängder och därmed rimligtvis bättre möjligheter till korrekta förutsägelser. På grund av osäkerheten i reaktionen på att få förutsägelsen gjordes experimentet i mindre skala med ett urval bestående av fem studenter. I kurs C tillfrågades de fem studenter med flest svarsförsök (mer än 150 försök vardera) på webbsidan. Av dessa fem valde fyra att genomföra undersökningen. Undersökningen ägde rum i den fjärde av kursens åtta läsveckor.

Experimentets utförande bestod i huvudsak av en kombination av en individuell semi-strukturerad intervju och ett tillitsformulär. För varje student var en undersökningsledare närvarande samt en observatör. Utförandet inleddes med en kontrollfråga vars syfte var att undersöka studentens förväntningar inför undersökningen. För att säkerställa att studenten var villig att ta del av sin förutsägelse ställdes även en kontrollfråga om studentens intresse. Förutsägelsen gavs till studenten via webbsidan YATA och ett exempel visas i Appendix \ref{app:AI-test-figur} varefter studenten fyllde i tillitsformuläret. Formuläret (se Appendix \ref{app:AI-test}) är en översatt version av ett tillitsformulär av \cite{trust} med vissa mindre ändringar för att passa situationen. Vårt tillitsformulär består av 14 påståenden där studenten fyller i hur väl personen håller med respektive påstående på en sjugradig skattningsskala. En etta motsvarar att deltagaren inte alls håller med påståendet och en sjua motsvarar att deltagaren fullständigt håller med påståendet. En fyra tolkas som ett neutralt svar. Medan studenten mottog förutsägelsen observerades studentens direkta reaktion. Utförandet avslutades med en semi-strukturerad intervju med mål att förstå studentens svar i tillitsformuläret samt den direkta reaktionen.

\begin{center}
\begin{figure}[hbtp]
    \centering
    \hspace{-20px}
    \resizebox {\textwidth} {!} {
        \definecolor{klight_green_400}{RGB}{156, 204, 101}

\tikzset{%
  project part/.style={
    rectangle,
    draw,
    fill=klight_green_400,
    thick,
    minimum width=3.2cm,
    minimum height=1.2cm
  },
  main line/.style={
    draw,
    line width=0.25mm,
    opacity=1,
    minimum size=1cm
  },
}

\begin{tikzpicture}[x=1.5cm, y=1.5cm, ->,>=stealth',auto, thick]
% Base project nodes
\node [project part/.try] (control) at (0,0) {$\textbf{Kontrollfråga}$};
\node [project part/.try] (predict) at (3,0) {$\textbf{Förutsägelse}$};
\node [project part/.try] (form) at (6,0) {$\textbf{Formulär}$};
\node [project part/.try] (interview) at (9,0) {$\textbf{Intervju}$};


% Connect them 
\path[main line/.style={font=\sffamily\small}]
    (control) edge[right] node [left] {} (predict)
    (predict) edge[right] node [left] {} (form)
    (form) edge[right] node [left] {} (interview);
\end{tikzpicture}
    }
    \caption{Flödeschema för experimentutförandet.}
    \label{fig:raket8}
\end{figure}
\end{center}

\section{Resultat och diskussion}
\label{sec: inter-RD}
Nedan presenteras och diskuteras resultatet ifrån interaktionerna. Först redogörs den observerade direkta reaktionen från studenterna på förutsägelserna. Därefter presenteras resultatet ifrån tillitsformuläret med tillhörande djupgående följdfrågor. Speciellt lägger vi fokus på gemensamma mönster hos testdeltagarna i analysen.

Gemensamt uttryckte deltagarna en nyfikenhet på att se förutsägelsen. En övergripande trend var även att det observerades en mindre anspänning inför förutsägelsen. Anspänningen noterades i testdeltagarnas ansiktsuttryck och av att tre av fyra deltagare uttryckte detta explicit vid den första kontrollfrågan. Emellertid var anspänningen på en hanterbar nivå och kan ytterligare ha påverkats av testsituationen. Vid förutsägelsen observerades vissa gemensamma beteenden i den direkta reaktionen. I huvudsak möttes förutsägelserna av en mindre förvåning, men generellt uppvisade studenterna neutrala känslor. De två studenter som gavs en förutsägelse som motsvarade underkänt i kursen uppvisade en större förvåningsreaktion än de två andra. En möjlig förklaring till den större reaktionen, som också uttrycktes av de två studenterna, är att förutsägelsen inte stämde överens med det personligt förväntade resultatet.

Vidare redovisas i figur \ref{fig:raket8} ett urval av testdeltagarnas svar på tillitsformuläret, se Appendix \ref{app:resultattillit} för samtliga påståenden. För att representera gemensamma mönster i tillitsfomuläret har vi valt ut påstående 1, 2 och 11, vilka var: ''Jag tycker förutsägelsen är trovärdig'', ''Jag tycker förutsägelsen är gjord på ett uppriktigt sätt'' respektive ''Förutsägelsen kan vara skadlig''.
\begin{figure}[hbtp]
    \centering
    \hspace{-35px}
    \resizebox {\textwidth} {!} {
        \definecolor{klight_green_200}{RGB}{197, 225, 165}
\definecolor{klight_green_300}{RGB}{174, 213, 129}
\definecolor{klight_green_400}{RGB}{156, 204, 101}
\definecolor{klight_green_500}{RGB}{139, 195, 74}
\definecolor{kgreen_300}{RGB}{129, 199, 132}
\definecolor{kgreen_500}{RGB}{76, 175, 80}
\definecolor{kgrey}{RGB}{222,222,222}
\definecolor{kblue}{RGB}{33, 150, 243}

% \pgfplotstableread[row sep=\\,col sep=&]{
%     interval & u & false \\
%     U     & 69  & 31 \\
%     3     & 83 & 17  \\
%     4     &    &     \\
%     5     &    &     \\
%     }\mydata

\begin{tikzpicture}
    \begin{axis}[
            ybar,
            x=7cm,
            enlarge x limits={abs=4cm},
            %enlarge y limits={abs=0.5cm},
            bar width=1cm,
            width=32cm,
            height=10cm,
            legend style={at={(0.5, 1.2)},
                anchor=north,legend columns=-1},
            legend image post style={scale=2},
            symbolic x coords={1, 2, 11},
            xtick=data,
            major x tick style = transparent,
            ymajorgrids = true,
            %nodes near coords={\pgfmathprintnumber[fixed,precision=0]{\pgfplotspointmeta}\,\%},
            nodes near coords align={vertical},
            ymin=0,ymax=7.5,
            yticklabel={\pgfmathparse{\tick}\pgfmathprintnumber{\pgfmathresult}},
            ylabel={\small Skattning},
            xlabel={\small Påstående},
            %ticklabel style = {font=\tiny},
            nodes={scale=2, transform shape}  % increase size of everything
        ]
        \addplot [fill=klight_green_200!100,draw=black!0.6] coordinates {(1,3) (2,2) (11,2)};  % Anv. 1
        \addplot [fill=klight_green_300!100,draw=kgrey!100] coordinates {(1, 1) (2, 6) (11, 1)};  % Anv. 2
        \addplot [fill=klight_green_400!100,draw=kgrey!100] coordinates {(1, 5) (2, 7) (11, 2)};   % Anv. 3
        \addplot [fill=klight_green_500!100,draw=kgrey!100] coordinates {(1, 1) (2, 7) (11, 4)};   % Anv. 4
        \addplot [fill=kblue!100,draw=kgrey!100] coordinates {(1, 2.5) (2, 5.5) (11, 2.25)};   % Medel
        \legend{Stud. 1, Stud. 2, Stud. 3, Stud. 4, Medel}
    \end{axis}
\end{tikzpicture}
    }
    \caption{Stapeldiagram för svaren på påstående 1, 2 och 11. Staplarnas höjd representerar studenternas (förkortat stud.) skattningar på den sjugradiga skalan. De fyra staplarna till vänster (olika nyanser av grönt) för varje påstående representerar deltagarnas svar och den femte till höger (blå) anger svarens medelvärde.}
    \label{fig:raket9}
\end{figure}
Syftet med påstående 1 är att undersöka deltagarnas upplevda trovärdighet för förutsägelsen. Medelvärdet 2.5 av 7 indikerar på ett lågt förtroende för förutsägelsen. Vid följdfrågorna om varför förtroendet var lågt var det genomgående svaret från samtliga deltagare att de trodde att det var för tidigt i kursen för att kunna göra en rimlig bedömning. Det fanns även en viss tveksamhet kring att deras aktivitet på webbsidan skulle kunna användas för att förutsäga deras tentaresultat. Två av deltagarna uppgav exempelvis att de hade använt facit när de löst en del uppgifter och pekade på att webbsidan inte kunde ta hänsyn till det. Dessutom noterar vi att de två deltagare som angav värdet ett på den sjugradiga skalan gavs en förutsägelse om ett underkänt resultat, vilket som tidigare nämndes om reaktionen inte stämde överens med deltagarnas egna förväntningar. En möjlig indikation är därför att trovärdigheten av förutsägelserna kan förbättras om de i en större utsträckning överensstämmer med mottagarens förväntningar. Den deltagare som angav värdet tre för påstående 1 uttryckte dock att förutsägelsen var i linje med personens förväntningar. Det relativt låga värdet tre pekar på att överensstämmelsen med den egna förväntningen inte är den enda avgörande faktorn för trovärdigheten, men även att faktorer som tidpunkt i kursen, användning och tillgång till facit har betydelse.
 
Trots lågt förtroende för förutsägelsen uttryckte studenterna att förutsägelsen var gjord på ett uppriktigt sätt i påstående 2. Medelvärdet var 5.5 av 7, där två av deltagarna fullkomligt höll med påståendet. En av deltagarna angav dock värdet två. Deltagaren motiverade värdet med att personen använt facit på flera av frågorna. Det kan därför vara nödvändigt att göra en distinktion mellan ärligheten i metoderna förutsägelsen baseras på och ärligheten i indatan till metoderna. Användningen av facit återkopplar även till trovärdigheten till förutsägelserna eftersom två av deltagarna uttryckte att facitanvändningen minskade deras förtroende. En möjlig lösning för att öka uppriktigheten och förtroendet kan därför vara att införa en facitfunktion på webbsidan som samlar in data över hur facit används.

Påstående 11 undersökte deltagarnas åsikter om möjlig skada förutsägelserna kan orsaka. Tre av fyra deltagare angav i tillitsformuläret att de ansåg att förutsägelserna hade små möjligheter att orsaka skada. Från följdfrågorna uttryckte de tre studenterna att de personligen inte såg någon skada med förutsägelsen, men att förutsägelsen kan leda till en ökad stress för andra studenter. Den fjärde studenten angav ett högre värde (en fyra) än de tre andra studenterna och motiverade skattningen med att personen ansåg att förutsägelsen skulle kunna medföra ökade stressnivåer, i synnerhet för studenter som precis har inlett sina universitetsstudier. Den upplevda stressen exemplifieras även av den femte tillfrågade studenten som valde att inte delta. Den femte studenten uppgav att den trodde att förutsägelsen skulle uppfattas som stressande och ville därför inte se den. Indikationen är således att den eventuella skada förutsägelsen kan orsaka för studenterna är en ökad stressnivå. Det är därför av vikt att utforma presentationen av förutsägelsen för att inte orsaka ökad stress.

En alternativ tillämpning för att minimera upplevd stress och öka trovärdigheten föreslogs av de fyra deltagarna oberoende av varandra. Grunden till förslaget uppkom från att varje deltagare undrade i fall tanken var att generera en ytterligare förutsägelse senare i kursen. Studenterna såg där en egen tillämpning av förutsägelsen. Tillämpningen var att låta förutsägelsen vara en återkommande återkoppling på hur de personligen ligger till i kursen. Till exempel svarade samtliga deltagare att de var intresserade att få se en ny förutsägelse i kursens sista vecka. De uppgav också att den förmodade stressen kunde minskas av att veckovis kunna se hur deras studier påverkade förutsägelserna. Den egna påverkan uttrycktes också av deltagarna kunna höja trovärdigheten för förutsägelsen. Exempelvis genom att kunna se upprepande gånger att deras ansträngningar får konsekvenser på förutsägelserna. Ett annat exempel innehållande upprepning uttrycktes av en av deltagarna, nämligen att trovärdigheten hade varit högre om personen kunde bedöma förutsägelsernas träffsäkerhet över flera kurser.

Vidare identifierar vi två huvudsakliga felkällor för undersökningen. Den första felkällan är antalet och urvalet av deltagare. Vilket har nämnts tidigare tillfrågades de fem studenter i kurs C med flest antal svarsförsök varav fyra valde att delta i undersökningen. På grund av det låga antalet deltagare anser vi att undersökningens indikationer ska bedömmas som kvalitativa. Dessutom medför urvalskriteriet att resultaten kan vara missvisande, exempelvis genom att de fyra deltagarna kan tillhöra en grupp av speciellt drivna studenter. Den andra felkällan är att deltagarna har valts från samma kurs varav det inte är undersökt hur väl indikationerna generaliserar till andra kurser. För att stärka indikationerna krävs därför en bredare studie med både fler studenter och från en större variation av kurser.

Sammanfattningsvis noteras följande för undersökningen av interaktionen mellan student och artificiell intelligens med utgång ifrån reaktions- och tillitsanalysen. Studenternas direkta reaktioner på förutsägelserna var neutral med en viss inledande anspänning. Tilliten delades in i framtagna komponenter som exempelvis trovärdighet och uppriktighet. Utifrån tillitsformuläret uttryckte studenterna att trovärdigheten för förutsägelsen var låg och uppriktigheten var hög. Deltagarna förmodade att en möjlig negativ påverkan av förutsägelsen kan vara en viss stressökning. Undersökningen indikerar alltså att de olika komponenterna till förutsägelsens tillit upplevs på olika nivåer. 

För att förbättra tilliten pekar undersökningen på att det finns en betydelse av att öka trovärdigheten av förutsägelsen. Till exempel observerades att förutsägelsens överensstämmelse med studenternas egna förväntningar och tillgång till facit påverkar trovärdigheten. För att öka trovärdigheten föreslog studenterna en möjlig lösning: att kontinuerligt återge förutsägelserna. Tillämpningen kan både ha en positiv effekt på förtroendet men även minska eventuell stresspåverkan.




