\chapter{Övergripande diskussion}
\label{sec:Disken}
Nedan följer en övergripande diskussion om arbetets gemensamma resultat. Diskussionen inleds med att sammanfatta arbetets syfte och huvudresultat vilka sedan utgör en grund för en fördjupande diskussion i avsnitt \ref{sec:future}. Specifikt resonerar vi utifrån resultaten kring vidare utvecklingsfunktioner på webbsidan YATA och den möjliga potential plattformen utgör. Därefter föreslår vi möjliga fortsatta studier och avslutar med en analys av samhälleliga och etiska aspekter i sektion \ref{sec:etik}. 

I introduktionen presenterades arbetets gemensamma syfte: att utveckla verktyg inom utbildning som förbättrar studenters potential att lära och lärares förmåga att undervisa. Med arbetet har vi tagit ett första steg för att uppfylla syftet genom den utvecklade plattformen bestående av webbsidan YATA med integrerad artificiell intelligens. För att utveckla plattformen har vi löst de tre delproblemen att: utveckla en flexibel och skalbar webbsida; undersöka möjligheten att utveckla en artificiell intelligens med förmåga att förutsäga studieresultat; undersöka interaktionen mellan student och artificiell intelligens.

Utvecklingen av webbsidan med fokus på flexibilitet och skalbarhet skedde i samverkan mellan insamlingen av behov och krav och den tekniska implementation av lösningen. Två resultat från behov- och kravanalysen var studenters behov av struktur samt deras krav på ögonblickliga funktioner. Dessutom identifierades en hierarki utifrån hur studenter söker hjälp när de fastnar i sina lösningsgångar. Upptäckten av hierarkin ledde till utvecklingen av en ögonblicklig hjälpfunktion vilken implementerades på YATA som en tipsfunktion. Andra implementationer än tipsfunktionen infördes iterativt på webbsidan och baserades även de på insamlad information. Flexibiliteten och skalbarheten åstadkoms genom valen att utveckla webbsidan med React och GraphQL. Webbsidan har i dagsläget använts av studenter i fyra olika matematiktunga kurser.

Med hjälp av djupinlärning lyckades vi utveckla en artificiell intelligens med god förmåga att förutsäga studieresultat. Till exempel uppnådde vi som mest en träffsäkerhet på $86 \pm 10\,\%$ med U/G-nätverket. Därtill presenterar vi i arbetet ett första steg med osäkerhetsnätverken för att modellera osäkerheter i de neurala nätverkens förutsägelser. En gemensam observation för U/G-nätverken och osäkerhetsnätverken är att det finns en förhoppning att förbättra träffsäkerheten ytterligare genom större datamängder.

Tillsammans utgjorde osäkerhetsnätverket och webbsidan YATA grunden för att undersöka interaktionen mellan student och artificiell intelligens. Webbsidan bidrog med data samt gav ett forum för studenterna att interagera med förutsägelserna, medan osäkerhetsnätverket producerade de faktiska förutsägelsena. Experimentets främsta indikation är att förutsägelsernas trovärdighet behöver ökas om tilliten ska förbättras. Ett möjligt förslag som identifierades för att öka trovärdigheten var att kontinuerligt återge uppdaterade förutsägelser.

\section{Fortsatt utveckling}
\label{sec:future}
Inledningsvis diskuteras möjliga funktioner att utveckla till webbsidan YATA med utgångspunkt i resultaten från kapitel \ref{sec: webb}-\ref{sec: inter}. Tillgång till facit var ett återkommande tema i flera av arbetets olika delar. Till exempel efterfrågades facit i både de första informationsinsamlingarna samt utvärderingarna efter iterationscykel 1 av webbsidan. Vi noterade även beteendet att vissa studenter skrev in facit i tipsfunktionen. Beteendet har inte utvärderats fullständigt, men indikerar att studenterna har ett stort behov av att kunna se facit när de räknar. I kapitel 4 diskuterades också att tillgång till facit på webbsidan kan förbättra förutsägelsens trovärdighet för studenterna, eftersom att facitanvändningen då skulle kunna användas som ytterligare indata till djupinlärningen.

En möjlig utökning av funktionaliteten på webbsidan är alltså implementationen av en facitfunktion. Däremot är det möjligt att invända att en facitfunktion kan ha eventuella pedagogiska nackdelar. Avsaknad av facit kan få studenterna att anstränga sig ytterligare för att lösa uppgifter eller få dem att söka hjälp av andra. Dock bidrar avsaknaden av facit till problematiken att studenter fastnar i lösningsgångar. Därtill drar sig många för att fråga om hjälp enligt den observerade hierarkin. För att implementera en facitfunktion finns det därmed flera krav. Utmaningen är att utveckla en facitfunktion som både möjliggör för studenterna att se facit, men som kräver en bestämd ansträngning innan facit tillgängliggörs. En ytterligare fördel, som beskrivits ovan, med en sådan facitfunktion är att funktionen kan höja trovärdigheten för förutsägelserna. Dessutom är det möjligt att insamlad data om studenternas facitanvändning kan förbättra djupinlärningsalgoritmerna.

På grund av webbsidan YATA:s flexibilitet är det möjligt att iterativt utveckla en effektiv facitfunktion enligt ovannämnda krav. Till exempel skulle facitfunktionen kunna kombineras med den redan framtagna tipsfunktionen. En första design kan vara att placera svaret ifrån facit sist i samlingen av tips sådant att studenterna först behöver läsa igenom tipsen innan de kommer fram till facit. Facitfunktionen kan även styras med att det krävs en viss ansträngning ifrån studenten innan den blir tillgänglig, vilket kan kvantifieras av exempelvis antalet försök samt den spenderade tiden på uppgiften.

En ytterligare observation som har gjorts i de tidigare kapitlen är förslag på bästa möjliga sätt att tillämpa förutsägelserna för studenterna. Som vi diskuterade i avsnitt \ref{sec: inter-RD} är en tillämpning att kontinuerligt återge förutsägelserna. Exempelvis skulle en veckovis återkoppling på hur studenterna ligger till i kursen kunna utformas. Designen kan representeras av till exempel en visare med de olika slutbetygen som mätvärden. I slutet av veckan kan då studenten se sina ansträngningar göra utslag genom att visaren ändrar position utifrån en förutsägelse. 

Vi identifierar att den främsta utmaningen för att åstadkomma en funktion med återkommande förutsägelser finns i djupinlärningen. Bedömningen baseras på att de utvecklade neurala nätverken i kapitel \ref{sec:Deep} inte är konstruerade för att vid olika tidpunkter generera förutsägelser. Ett möjligt tillvägagångssätt är att utveckla specifika nätverk för varje läsvecka sådant att veckovisa förutsägelser kan framställas. Notera att ju färre veckor av kursen som passerat desto mindre data finns tillgängligt för nätverken, vilket försvårar utmaningen. Vi bedömmer ändå att det är möjligt utifrån ett djupinlärningsperspektiv att utveckla en funktion med återkommande förutsägelser, speciellt vid tillgång till större datamängder.

Vidare argumenterar vi för nyttan och potentialen med en fortsatt utveckling av plattformen. Till att börja med har vi identifierat att det finns ett potentiellt värde med plattformen för både studenter, lärare och institutioner i helhet. För studenter finns möjligheter att underlätta och effektivisera studierna, förbättra inlärningen och minska arbetsbelastningen. En smidig och lättåtkomlig applikation kan underlätta och effektivisera studierna genom att stödja studenten med en bättre struktur. En flexibel webbsida som YATA möjliggör att iterativt utveckla nya pedagogiska funktioner, som exempelvis tipsfunktionen och den föreslagna facitfunktionen, vilka kan förbättra inlärningen. Förutsägelser om studieresultat kan dessutom ge studenter möjlighet att anpassa sina studier i tid. Kontinuerliga förutsägelser har även potential att stimulera studenten till att arbeta mot ett mål. Kontinuiteten kan också minska studenters stress och oro då de ser att deras ansträngningar ger resultat och de får en bekräftelse på att de kommer nå sina målsättningar. 

Plattformen kan utvecklas för att förbättra lärares förmåga att undervisa. Datainsamlingen kan utöver att användas till djupinlärningsalgoritmerna också användas för att återkoppla information till läraren. Informationen kan exempelvis bestå av statistik över hur det går för studenterna i deras arbete. Därtill skulle läraren kunna uppmärksammas på uppgifter som studenter har svårigheter med. Den beskriva återkopplingen möjliggör för lärare att dynamiskt justera sin undervisning för att bättre undervisa studenterna.

Vi vill därtill betona vikten av att undersöka de ovannämnda förslagen iterativt. Det är möjligt att invända att flera av idéerna kan medföra negativa pedagogiska effekter. Studententernas noterade stressproblematik är en negativ aspekt som kan påverka inlärningen. Dessutom kan återkopplingen av information till lärare öka den administrativa belastningen. Vi rekommenderar därför en balansgång mellan viljan att förbättra och riskerna med att orsaka negativ påverkan. För ett vidare resonomang om balansgången hänvisar vi till diskussionen om samhälleliga och etiska aspekter i avsnitt \ref{sec:etik}.

Vi noterar även ett möjligt värde för institutionen som ansvarar för utbildningen. Tidiga och träffsäkra förutsägelser av studieresultat möjliggör för tidiga åtgärder vid behov. Dessa insatser kan göra att det totala antalet underkända studenter minskar. En möjlig följd är en bättre genomströmning av studenter som genomför sin utbildning på utsatt tid.

Slutligen föreslår vi möjliga arbetsområden för fortsatta studier och utveckling. Webbsidan YATA kan utvecklas på flera olika sätt. Vi ser i huvudsak två möjliga inriktningar med stort potentiellt värde. Dessa är den beskrivna facitfunktionen och en funktion för att kontinuerligt kunna ge förutsägelser. Som även lyftes i avsnitt \ref{sec: webb-D} behöver svarshanteringen vidareutvecklas genom bättre direkt återkoppling till användaren. I fortsatta studier ser vi också ett behov av att analysera lärarens behov och krav för att utvidga YATA:s målgrupp.

För att generellt utveckla de framtagna neurala nätverken ser vi primärt tre utvecklingsområden: förbättra träffsäkerheten, undersöka det identifierade generaliseringsproblemet (se avsnitt \ref{sec:Deep-D}) och vidare modellera osäkerheterna i förutsägelserna. Gemensamt för de tre områdena är kravet på större datamängder.

Därtill kan den kvalitativa undersökningen av interaktionen mellan student och artificiell intelligens utvecklas. Den främsta bristen i undersökningen är storleken på antalet deltagare i kombination med att samtliga deltagare var studenter i samma kurs. För att stärka indikationerna behöver idealt en större kvantitativ studie genomföras med studenter från flera olika kurser.


%Nyttan av arbetet kan diskuteras utifrån arbetets syftesformulering. Specifikt vill vi problematisera kring formuleringen ’’förbättrar studenters potential att lära och lärares förmåga att undervisa’’. Vi anser att begreppet förbättra inte är väldefinierat i arbetet. Formuleringen att ’’förbättra studenters potential att lära’’ kan tolkas utifrån flera perspektiv. Till exempel att studenter lär sig med en djupare förståelse, snabbare eller med mindre stress. Liknande kan formuleringen ’’förbättra […] lärares förmåga att undervisa’’ innehålla tolkningar som en minskad arbetsbelastning och möjligheter till rättvisare bedömningar av studenterna. Vår tolkning inom arbetets kontext av begreppet förbättra är att begreppet förbättra innehåller olika dimensioner såsom djupare förståelse, snabbare inlärning och mindre arbetsbelastning. Dimensionerna utgör olika frihetsgrader som kan motsäga varandra och kan därför behöva optimeras. I vidare arbete är det härav nödvändigt att vara medveten om de olika dimensionerna för att sammanlagt arbeta mot en gemensam förbättring.

% Eftersom projektet omfattar direkt kontakt med människor, främst i egenskap av studenter och lärare, finns det samhälleliga- och etiska aspekter som behöver beröras. En negativa aspekt är insamlingen av individers data. Beroende på vilken typ av data som samlas in kan individers integritet påverkas. Exempelvis är det skillnad på att samla in studenters svar på räkneuppgifter och extremfallet att använda kameror för att filma individer medan de studerar. För att minimera integritetskränkande insamling i utförandet krävs därför att rådande lagstiftning för personuppgifter i form av Datorskyddsförordningen (GDPR) följs och att samarbeten sker utifrån frivillighet.

% Ett annat möjligt etiskt problem är användandet av artificiell intelligens i form av djupinlärning för att förutsäga studieresultat. Om individer informeras under en kurs om deras respektive förutsägelser kan det medföra negativa konsekvenser. Till exempel kan ett informerande om att en elev förutsägs åstadkomma ett underkänt resultat leda till att elevens motivation minskar. Ett alternativ för att minimera negativ inverkan är därför att utföra de första testerna på små grupper i storleksordning 5-10 personer som är fullständigt införstådda med testernas syften och risker.

% Däremot kan resultatet för förutsägelser av studieresultat även innebära positiva samhälleliga- och etiska aspekter. Analogt med att negativa besked kan medföra en minskad motivation kan det medföra en möjlighet att genomföra förändringar innan ett negativt resultat åstadkoms. Det är också möjligt att en tidig indikation på ett underkänt resultat kan leda till en ökad motivation. En på längre sikt möjlig positiv konsekvens är att artificiell intelligens med förmåga att förutsäga betyg med hög noggrannhet kan vara ett kvantitativt verktyg som möjliggör för lärare att sätta rättvisare betyg. 

% Ytterligare kan slutresultet för förutsägelser av studieresultat medföra samhällelig påverkan och etiska aspekter. Om det är möjligt att via algoritmer baserade på Deep Learning att förutsäga studieresultat med hög tillförlitlighet är ett eventuellt scenario att det implementeras på stor skala. En möjlig konsekvens är att stora grupper med elever som får negativa besked påverkas negativt exempelvis genom att deras motivation att fortsätta studera minskar. På samma sätt kan elever som får ett positivt resultat uppvisa en nöjdhet och därför slutar med att forsätta anstränga sig för att vidareutvecklas.

% Avslutningsvis medför ovan exempel att både arbetets genomförande och resultat kan innebära negativa och positiva samhälleliga- och etiska aspekter. Det är därför av vikt att medvetet minimiera eventuella negativa konsekvenser. Vi anser dock utifrån ovan möjliga positiva konsekvenser att projektet påvisar en större eventuell nytta än möjlig skada, vilket medför utifrån ett etiskt perspektiv att projektet bör genomföras.

% Struktur
    % Beskriv problemet - vi gör förutsägelser på människor
    % Datainsamling ?
    % Vad kan förutsägelsen resultera i - använd resultatet från AI-testet
        % Negativa aspekter - Stress, osäkerhet i förutsägelsen
        % Positiva aspekter - Kan ges möjligheten att förbättra sitt resultat
            % Förbättringspunkt - Ge kontinuerlig bedömning, ger studenterna möjlighet att bättra sig
    % Vidga perspektivet - Samhälleliga konsekvenser 
    % Slutsats - Arbetet borde fortsätta, då det finns fördelar. Dock behöver studier på människor fortsätta vara centrala för att undvika negativ påverkan.

\section{Samhälleliga och etiska aspekter}
\label{sec:etik}

% Intro - balansgång i det större etiska och samhälleliga perspektivet
% Integritet - som det är
% Nackdelar:
    % Stress - som observerades är det den främsta nackdelen våra tillfrågade ser. 
    % (Administrativ börda)
% Fördelar
    % Visshet - vet hur man ligger till
    % Jämnare betygssättning
    % Förbättra lärares arbetssituation och utlärning
    % På lång sikt: ekonomiska och resurs-vinster i att elever lär sig bättre och effektivare
% Sammanfattning - fördelrna överväger nackdelarna men viktiga att ha i åtanke. 

I detta arbete men även i fortsatta studier identifierar vi att det finns en balansgång mellan viljan att förbättra och riskerna med att orsaka negativ påverkan. I följande avsnitt samlar vi de delar av arbetet som främst har en påverkan på samhälleliga och etiska aspekter. Dessa delar är: datainsamling, förutsägelse av studieresultat och pedagogiska verktyg. 


Ett viktigt perspektiv är studenternas integritet. Datainsamling över studenters studiemönster är nödvändig för att generera förutsägelser, men kan även påverka deras integritet. För att undvika att påverka integriteten för de studenter vars data har använts i detta arbeta har datan samlats in i enlighet med rådande lagstiftning för personuppgiftslagring i form av Dataskyddsförordningen (GDPR). Bland annat innebär det att användardata som använts för att träna nätverken har anonymiserats före åtkomst och att användandet av webbsidan YATA har skett på frivillig grund. 

Förutsägelserna av studieresultat kan ge upphov till negativ påverkan både om beskedet är positivt eller negativt. En negativ förutsägelse kan som noterades i kapitel \ref{sec: inter} ge upphov till stress för studenterna. Dessutom är det möjligt att en positiv förutsägelse kan leda till att studenter invaggas i en falsk trygghetskänsla. Utifrån ett etiskt perspektiv är det problematiskt om förutsägelserna påverkar studenters välmående negativt. Därtill kan det finnas en samhällelig påverkan på längre sikt. Exempelvis om förutsägelserna implementeras på stor skala är det viktigt att de är rättvisa. Risken är annars att förutsägelserna kan bedöma olika grupper av studenter på olika sätt. 

Förutsägelserna kan även ha flera positiva aspekter. I kapitel \ref{sec: inter} noterades utöver en möjlig utökad stress att förutsägelserna kan minska studenternas stress och oro, speciellt om förutsägelserna återkopplas kontinuerligt. En negativ förutsägelse kan också ha en positiv etisk inverkan genom att studenterna motiveras till att göra förändringar i sina studier. Ur ett samhälleligt perspektiv kan förutsägelserna, om de blir tillräckligt tillförlitliga, agera som ett databaserat bedömningsstöd för lärare, vilket skulle kunna leda till rättvisare bedömning. Dessutom kan förutsägelserna användas av institutioner som skolor och universitet för att identifiera behov av insatser. En möjlig positiv följd av sådana insatser är en ökad genomströmning av studenter. För samhället medför en ökad genomströmning att utbildningsresurser används mer effektivt samt att den tillgängliga kompetensen från utbildade personer ökar. 

Vi noterar att förutsägelserna kan ha både positiv och negativ inverkan utifrån ett samhälleligt och etiskt perspektiv. Ambivalensen återkommer för pedagogiska verktyg i allmänhet och webbplattformen YATA i synnerhet. Ett exempel som tidigare har tagits upp är den implementerade tipsfunktionen. Tipsfunktionen är ett försök att lösa problemet att studenter kör fast när de löser uppgifter, med den etiska vinsten att studenternas inlärning förbättras. Däremot kan tipsfunktionen försämra inlärningen genom att studenter inte lägger ner den ansträning som krävs för en ordentlig inlärning. Dessa påverkningar på inlärningen gäller på samma sätt för den föreslagna facitfunktionen. Ett ytterligare förslag som har presenterats är att YATA kan användas för att återkoppla statistik över studenternas studier till lärare. Det ger möjligheten för lärare att utifrån informationen anpassa och förbättra sin undervisning. På stor skala skulle det kunna innebära att utbildningsresurser utnyttjas mer effektivt. Nackdelen är att återkopplingen kan medföra en ökad administrativ börda för lärare.


% Ambivalens 
% Balansgång -> iterativt arbete
% Medvetenhet om påverkan och nyttan med det -> motiverar fortsatt arbete

De ovan beskrivna aspekterna har olika karaktär. Datainsamlingen måste ske enligt ett lagstadgat regelverk, vilket ger lite utrymme för tolkning. De samhälleliga och etiska aspekterna är däremot mindre entydiga för förutsägelserna och pedagogiska verktyg som YATA. Beroende på hur förutsägelserna används och hur webbplattformen YATA utformas kan de både ha en positiv och negativ påverkan. I det fortsatta utvecklingsarbete finns det därmed en balansgång mellan att förbättra och riskerna att förvärra. För att minimera riskerna behöver det fortsatta arbetet ske iterativt. Med ett iterativt arbetssätt och en medvetenhet om aspekterna ovan motiverar den potentiella nyttan en fortsatt utveckling av arbetet.
