\chapter{Slutsats}
%En webbplattform har utvecklats för utbildning bestående av en flexibel och skalbar webbsida med integrerad artificiell intelligens. Webbplatformen utgör en del av en framväxande digitalisering av utbildning i kombination med den tekniska utvecklingen av tillämpad artificiell intelligens. Den framtagna webbsidan YATA utgör grunden till webbplattformen. Till exempel möjliggör webbsidan möjlighet att utveckla nya pedagogiska verktyg som den framarbetade tipsfunktionen. Tipsfunktionen baserades på den observerad hierarkin i studenternas tillvägagångssätt för att söka hjälp. 

%Webbsidan YATA genom sin datainsamling möjliggör även utvecklingen av artificiell intelligens baserad på djupinlärning för att förutsäga studieresultat. Med hjälp av djupinlärning har neurala nätverk utvecklats för att förutsäga kursbetyg och skrivningspoäng på universitetsnivå. Med våra utvecklade nätverk uppnås exempelvis en träffsäkerhet på 80-90 \% för att förutsäga underkänt eller godkänt resultat i två matematiktunga kurser. Därtill har vi föreslagit våra osäkerhetsnätverk som en första möjlighet att modellera osäkerheter i nätverkens förutsägelser. Med större datamängder förväntas resultaten att kunna förbättras ytterligare.

%En första interaktion mellan student och artificiell intelligens har undersökts genom en kvalitativ studie. Studiens resultat var att studenterna har lågt förtroende för förutsägelserna, men att de upplever uppriktigheten i förutsägelserna som hög. De fyra deltagande studenterna noterade att förutsägelsen kan medföra en risk att verka stressande. Studenterna identifierade även att en kontinuerligt återkommande förutsägelse skulle kunna höja trovärdigheten för förutsägelsen.

%Avslutningsvis identifierar vi en vidare utvecklingspotential av webbplattformen. Webbplattformen utgör ett första steg i att kombinera webbteknik med artificiell intelligens för att förbättra studenters potential till att lära och lärares förmåga att undervisa.

Med en användarcentrerad utveckling har webbsidan YATA tagits fram med fokus på flexibilitet och skalbarhet. Flexibiliteten har möjliggjort en iterativ utvecklingsprocess vilken har resulterat i en insiktsfull informationsmängd kring studenternas behov och krav. Exempelvis observerades en hierarki i hur studenter söker hjälp. Nya koncept har tagits fram utifrån informationsmängden för att underlätta för studenter. Från koncepten har vi utvecklat webbsidan med en tydlig översikt och struktur samt en tipsfunktion som motverkar problemet att studenter kör fast i lösningsgångar. I dagsläget har studenter i fyra olika kurser använt webbsidan YATA utan incitament och hjälpt varandra genom tipsfunktionen.

Webbsidan YATA möjliggör genom datainsamling en utveckling av artificiell intelligens baserad på djupinlärning för att förutsäga studieresultat. Specifikt utvecklades neurala nätverk för att förutsäga kursbetyg och skrivningspoäng på universitetsnivå. Våra nätverk uppnår exempelvis en träffsäkerhet på $86\pm 10\,\%$ respektive $84\pm 7\,\%$ för att förutsäga underkänt eller godkänt resultat i två matematiktunga kurser. Därtill har vi utvecklat osäkerhetsnätverk som ett första steg för att modellera osäkerheter i nätverkens förutsägelser. Med större datamängder förväntas resultaten kunna förbättras ytterligare.

Kombinationen av YATA och den framtagna artificiella intelligensen utgör tillsammans en webbplattform för utbildning. Med denna plattform kunde vi undersöka en första interaktion mellan student och artificell intelligens genom en kvalitativ studie. Vi fann att studenterna har lågt förtroende för förutsägelserna, men att de upplever uppriktigheten i förutsägelserna som hög. De fyra deltagande studenterna noterade att förutsägelsen kan medföra en stressrisk. Deltagarna föreslog dessutom en vidareutveckling för att höja trovärdigheten och potentiellt minska stressrisken. Föreslaget var att kontinuerligt återge uppdaterade förutsägelser.

Utifrån användningen av webbsidan och det inneboende värdet i förutsägelserna argumenterar vi för att det finns en stor potential för vidareutveckling. Plattformen utgör ett verktyg för pedagogiska funktioner såsom tipsfunktionen och den föreslagna facitfunktionen. Plattformen möjliggör också undervisningsanpassning genom att återkoppla information till läraren, exempelvis kring vilka uppgifter studenterna fastnar på. Därtill kan veckovisa uppdaterade förutsägelser ge studenterna en inblick i hur de ligger till och erbjuda dem ett sätt att se hur deras ansträningar påverkar studierna. Plattformen utgör därför ett första steg för att förbättra studenters potential att lära och lärares förmåga att undervisa.
