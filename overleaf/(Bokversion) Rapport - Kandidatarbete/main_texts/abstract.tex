\noindent YATA\\
Webbplattform för utbildning med integrerad artificiell intelligens\\
Simon Pettersson Fors, Marcus Holmström, Anton Hägermalm, Eric Lindgren, Carl Nord, Gabriel Wallin, Arianit Zeqiri\\
Institutionen för fysik\\
Chalmers Tekniska Högskola\\


\vspace*{1cm}
\begin{flushleft}
    \Large
\noindent \textbf{Abstract}
\end{flushleft}



\noindent Web platforms are a common occurrence in education to try to improve student results. However, recent effects of digitalization and breakthroughs in the field of artificial intelligence have revealed an uncharted potential in the combination of education, web technology and artificial intelligence. In this thesis, we propose a flexible and scalable website incorporating an integrated artificial intelligence with the ability to predict university student results as a first attempt to explore the described potential. Here, we describe the development of the website and artificial intelligence as well as an experiment which investigates the interaction between student and artificial intelligence. We argue that our developed web platform could form a basis for further studies and give several suggestions for the future development of this promising educational tool. 

\vspace{2.5cm}
\begin{flushleft}
    \Large
\noindent \textbf{Sammandrag}
\end{flushleft}


\noindent Webbplattformar är vanligt förekommande inom utbildning för att försöka förbättra studieresultat. De senaste årens effekter av digitalisering samt genombrott inom artificiell intelligens tyder emellertid på en outforskad potential i kombinationen av utbildning, webbteknik och artificiell intelligens. Som ett första försök att utforska den beskrivna potentialen, föreslår vi en flexibel och skalbar webbsida med en integrerad artificiell intelligens med förmågan att förutsäga universitetsstudenters  studieresultat. Vi beskriver utvecklingen av webbsidan och artificiell intelligens samt ett experiment som undersöker interaktionen mellan student och artificiell intelligens. Vi argumenterar för att den utvecklade webbplatformen kan ligga till grund för ytterligare studier och föreslår flera inriktningar av framtida utveckling av utbildningsverktyget.

%\vspace{4.5cm}

%\noindent Keywords: Edtech, education, artifical intelligence, deep learning, university