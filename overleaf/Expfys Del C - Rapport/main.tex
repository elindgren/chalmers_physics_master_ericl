\documentclass{article}
\usepackage[utf8]{inputenc}

\title{Expfys Del C - Rapport}
\author{forssi student}
\date{April 2018}

\begin{document}

\maketitle

\section{Metod}

\subsection{Försöksuppställning}


För att undersöka ett eventuellt kaotiskt beteende för den dubbla koniska pendeln användes samma försöksuppställning som för fastställandet av g, med tillägget av en extra pendel i änden av stången som var fastsatt i motorn. Se figur xx. I ändpunkterna av de två stängerna placerades IR-markörer för att båda punkterna skulle gå att följa.

\subsection{Utförande}

Då ett kaotiskt system är synnerligen känsligt för initialvillkor var dessa viktiga att bestämma för att uppmätt data skulle kunna jämföras mot motsvarande datorsimulering. Detta gjordes genom att fixera vinkelförhållandet mellan den övre och den undre stången med ett snöre fäst mellan deras ändar; en mot motoraxeln (överst i systemet) och en under IR-markören på den undre pendeln (underst i systemet). Med vinklarna fixerade användes DC-motorn för att få systemet att uppnå ett jämviktsläge med ungefär de initialvinklar och den rotationshastighet som skulle undersökas. Exakta vinklar och rotationshastighet bestämdes genom att göra en kortare mätning med IR-kamorerna ur vilken dessa kunde härledas trigonometriskt. 

Därefter startades en ny, längre mätning och snörändan fäst mot motoraxeln lossades. Motoraxeln valdes då eventuella krafter där ej påverkar systemet. Den undre pendelns beteende dokumenterades av kamerorna, och den uppmätta datan jämfördes mot systemets simulerade beteende med de initiallvillkor som bestämdes under mätningen med fixerade vinklar. 



\end{document}
