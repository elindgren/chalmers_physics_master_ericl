%\documentclass[12pt,a4paper]{article}
%\usepackage[left=2.5cm,top=2.5cm,right=2.5cm,nofoot]{geometry}
%\usepackage{graphicx}
%\pagestyle{myheadings}
%\usepackage[utf8]{inputenc}   
%\usepackage[T1]{fontenc}      
%\usepackage[swedish]{babel}
%\usepackage{lmodern}
%\usepackage[export]{adjustbox}
%\usepackage{amsmath}
%\usepackage{amssymb}
%\usepackage{graphicx}
%\usepackage{units}
%\usepackage[]{gensymb}
%\usepackage{float}
%\usepackage{hyperref}
%\usepackage{wrapfig}
%\usepackage{array}
%\usepackage{fancyhdr}
%\usepackage{ stmaryrd }
%
%\usepackage{caption}
%\usepackage{subcaption}
%\usepackage{cleveref}
%\usepackage{appendix}
%\usepackage{csquotes}
%\usepackage{ragged2e}
%\usepackage{pdfpages}
%\usepackage{textcomp}
%\usepackage{xcolor}
%\usepackage{wrapfig}

%\begin{document}

%\title{Peer review - Respons och åtgärder}

%\author{Simon Pettersson Fors (forssi), Eric Lindgren (ericlin)}

%\maketitle

\subsection{Peer-review-kommentarer}
Observera att nedan kommentarer är ordagrant citerade varav stavfel och liknande ej har korrigerats. 

(1) Välidgt bra rubrik för en rapport- Ni tar kort upp var huvuduppgiften kommer att handla om och vilka termer som undersöks.
\newline
(2) Ett bra sammandrag, inte mycket att lägga till då en kortfattad presentation på själva uppgiften och en disskution på hur laborationen har för betydelse  efterhand.
\newline
(3) Dela gärna upp Metod till Teori och Metod. Ser även kommentar under rubriken "Metod" i sid 2.
\newline
(4) En väldigt övergripande inledning och en tydligt koppling till uppgiften som ni har utfört. Ni har även delat upp inledningen i stycken vilket gör det enkelt för läsaren att följa texten.
\newline
(5) Antar att detta är den extrauppgift ni tänkt er?
\newline
(6) Ni gör en mix av metod och teori under denna punkt, vilket gör det ganksa svårt till att veta när ni går till själva utförandet av laborationen. 
\newline
(7) Ganska svårt att se bildens relevans under själva utförandet. Om ni har möjlighet, visa även en 3D bild i själva systemet. Ni kan även beksriva  uppställningen lite tydligare.
\newline
(8) En smaksak, men väldigt bra att ni även har en inledning för själva resultatet och diskussionen.
\newline
(9) Väldigt bra att ni utvecklade er beskrivning för figur 3a)s utseende.
\newline
(10) Detta är väldigt många mätningar, och man måste fråga om ni hade lika många mätningar i varje mätserie(vilket ni ej hade) och ni i princip tog ut flera mätvärden för en (mätning). Det kan vara ganska att misstolka detta, om ni ej hade visa mätning 13 sedan innan.
\newline
(11) Ni menar väl i funktion av mätperiod. Kan tolkas som att ni har haft pendeln igång i en väldigt lång stund.
\newline
(12) Ni kan även förklara kortfattat till vaför värdet på g minskar grovt under mätnig 2000 och 2300, för att sedan återgå till det gamla, en felkälla?
\newline
(13) Hör till figurkommentaren, men ni bör även har markerat i grafen på var mätningarna från mätning 13 uppkommer i grafen.
\newline
(14) Väldigt bra att ni tar upp Tröghetsmatriserna i appendix, och inte i teoridelen, då det inte har en större relevans i själva grunduppgiterna ni utför
\newline
(15) Antar att ni har glömt att lägga in matlabkoderna och labbloggen. Lägg gärna till dem.



\subsection{Åtgärder för kommentarer från Peer-review}

\noindent (3) \& (6) Vi valde att inte dela upp Metodavsnittet i en Teori- och en Metod-del då vi anser att det ger en tydligare röd tråd.
\newline

\noindent (7) Bilden av kardanknuten har gjorts tydligare genom en omarbetning i ritprogrammet Inkscape. 3D-bild över systemet visas i figurer på sidan innan, det som inte finns med är en mer övergripande skiss över försöksuppställningen men en sådan anses överflödig då uppställningen redan beskrivs i texten i kombination med de bilder som redan finns.
\newline 

\noindent (10) Detta har förtydligats genom att tydligare förklara hur mätningarna och beräknandet av $g$ har gått till.
\newline

\noindent (11) Pendel var igång upp till 200 sekunder vilket redogörs för i avsnitt \textit{Utförande för bestämning av $g$}.
\newline

\noindent (12) Detta är en bra synpunkt, men vi har ännu inte lyckats få in detta på ett bra sätt.
\newline 

\noindent (13) Detta har nu gjorts.


%\end{document}