\chapter{Introduktion}
\label{sec:intro}

Webbplattformar har varit ett vanligt verktyg för lärare och studenter sedan internets introduktion inom utbildning. Nya möjligheter har dock vuxit fram i och med de senaste årens transformerande effekter som följd av digitaliseringen av media. Effekter som den ökade tillgängligheten och demokratiseringen av akademiskt innehåll, samt överskottet av material från otaliga användare har skapat nya sätt för människor att dela och ta till sig ny kunskap. Framväxten av \emph{Edtech-industrin} (eng. \emph{Educational Technology}) med stora öppna onlinekurser, mer informella mobila inlärningsapplikationer och olika peer-to-peer lösningar är exempel på nya sätt att lära som internet har möjliggjort. Tillsammans med de senaste årens framgångar inom artificiell intelligens, specifikt djupinlärning som utnyttjar de stora datamängder som dessa webbplattformar genererar, gör det kombinationen av utbildningsteknologi och AI intressant. Intresset speglas bland annat av att läromedelsföretag och stora internationella aktörer såsom Google gör stora satsningar inom industrin \cite{Metaari}. Trots den betydande tekniska utvecklingen är de program som används i traditionell undervisning oftast undermåliga och ofullgångna \cite{Frodin}. Detta medför att det finns en stor potential i att utveckla nya, moderna verktyg för användning i skolor och på universitet. Verktyg som är anpassade efter den samtida tekniska utvecklingen samt efter studenternas och lärarnas behov. 

På universitetsnivå finns flera internetbaserade verktyg till studenters och kursansvarigas förfogande i matematik och beräkningstunga ämnen. Några exempel är \emph{MapleTA} \cite{MapleTA}, \emph{Piazza} \cite{Piazza} och det av Göteborgs universitet utvecklade \emph{OpenTA} \cite{OpenTA}. I utvärderingar av dessa program (se Appendix \ref{app:opentasurvey}) framkommer indikationer på vad studenter värdesätter med de webbaserade verktygen. En viktig aspekt är den översikt och struktur programmen ger vilket kan underlätta studierna. Dessutom visar utvärderingarna på ett vanligt förekommande problem i inlärningen i matematiktunga kurser. Inom sådana kurser är det vanligt att en stor andel studenter fastnar under lösningsgången av räkneuppgifter och som en konsekvens efterfrågar lösningsförslag och tips för att bättre förstå uppgiften.

När studenter interagerar med webbplattformar uppstår även möjligheten att generera data över studenternas studiemönster. En potentiellt värdefull användning av datan är att förutsäga studenters studieprestationer. För studenten innebär det möjligheten att få tidig information om det sannolika slutresultatet varpå eventuella förbättringar kan genomföras tidigt för att uppnå ett önskat utfall. På samma sätt kan lärare få information om nivån på studenteras kunskaper och anpassa undervisningen därefter. På en högre organisationsnivå kan även skolan eller institutionen, vilka är ytterst ansvariga för utbildningen, värdesätta förutsägelserna eftersom informationen möjliggör tidiga insatser.

Djupinlärning (eng. \emph{Deep Learning}) är en möjlig metod för att utveckla system med förmåga att förutsäga studieresultat. Till exempel har djupinlärning tillämpats på stora öppna onlinekurser (eng. \emph{Massive Open Online Courses}) med goda resultat \cite{Kim2018GritNetSP}. Dessa kurser erbjuder både stora datamängder, vilket ofta är nödvändigt för djupinlärning, och ett binärt studieresultat i form av ett godkänt eller underkänt betyg. Det är däremot ej undersökt om resultaten generaliserar till matematiktunga universitetskurser som också har mer högupplösta studieresultat i form av graderade betyg och skrivningspoäng. Ytterligare tar inte tidigare arbeten hänsyn till osäkerheten i förutsägelserna.

Utifrån ovan beskrivna möjligheter identifierar vi ett behov av att utveckla en lättföränderlig webbsida. En sådan flexibel webbsida utgör ett verktyg för studenter och lärare som underlättar deras arbete. Till exempel kan webbsidan användas för att försöka åtgärda inlärningshindret som uppstår när studenter fastnar i sina lösningsgångar. Ytterligare möjliggör det datainsamling för att förutsäga studieresultat med hjälp av djupinlärning samt en kanal för att återkoppla dessa förutsägelser till användarna.

Dessutom kopplar en användacentrerad utveckling av webbsidan till insamlingen av data. Även om det finns en webbsida med stöd för att samla in data är det väsentligt att studenterna använder plattformen för att data ska genereras. Om studenterna frivilligt ska använda webbsidan krävs det att verktyget erbjuder ett mervärde för användaren. Det är därför nödvändigt att analysera och ta hänsyn till studentens behov och krav för att kunna erbjuda ett mervärde och öka användningen.

Ytterligare utmaningar uppstår i kontakten med studenterna. Det är okänt hur studenter reagerar på en artificiell intelligens med förmåga att förutsäga deras studieresultat. Studentens motivation kan påverkas negativt av en negativ förutsägelse och studenten kan även invaggas i en falsk trygghetskänsla vid ett positivt besked. Däremot kan förutsägelsen ha positiv påverkan genom att studenten tidigt i inlärningsfasen får information om att en förändring av dess studier borde genomföras. För att förutsägelsen ska medföra ett värde krävs alltså att interaktionen mellan student och artificiell intelligens undersöks.

%För att utveckla nya verktyg inom utbildning som förbättrar studenters förmåga att lära och lärares förmåga att lära ut utvecklade vi en ny flexibel och skalbar webbsida med integrerad artificiell intelligens med förmåga att förutsäga studieresultat. Här presenteras utvecklandet av webbsidan, analysen av metod- och designval för plattformen samt framtagningen av en artificiell intelligens baserad på djupinlärning med syfte att förutsäga studieresultat. Dessutom redovisas en första undersökning av interaktionen mellan studenter och artificiell intelligens under en pågående utbildning.

\section{Syftes- och målformulering}
\label{sec:purpose_and_problem}
För att utveckla verktyg inom utbildning som förbättrar studenters potential att lära och lärares förmåga att undervisa har arbetet två mål. Det första målet är att utveckla en flexibel och skalbar webbsida med integrerad artificiell intelligens med förmåga att förutsäga studieresultat. Det andra målet är att undersöka interaktionen mellan student och artificiell intelligens under en pågående utbildning.

\section{Problemformulering och avgränsningar}
För att uppnå ovan beskrivna syfte delar vi in arbetet i följande tre delproblem. Det första delproblemet är att utveckla en flexibel och skalbar webbsida som underlättar arbetet för studenter och lärare och som integrerar en artificiell intelligens för att förutsäga studieresultat. Utöver den tekniska utvecklingen är det nödvändigt att analysera användarnas behov och krav för webbsidan. Analysen av behov och krav krävs för att definiera den grundfunktionalitet webbsidan behöver erbjuda. Specifikt studeras funktionalitet för att lösa problematiken att en stor andel studenter fastnar i lösningsgångar.

Det andra delproblemet är att undersöka om det är möjligt att utveckla en artificiell intelligens baserad på djupinlärning med förmåga att förutsäga studieresultat för matematiktunga kurser på universitetsnivå. Vi reducerar problemet till följande delar: att utföra en statistisk analys på en tillgänglig datamängd för studenters studiemönster, att utveckla algoritmer baserade på djupinlärning för att förutsäga studieresultaten underkänt eller godkänt betyg, graderat betyg och skrivningspoäng samt att analysera när algoritmerna utför felaktiga förutsägelser och utreda orsakerna till felen. Dessutom undersöks möjligheten att modellera osäkerheten i algoritmernas förutsägelser. 

Det tredje delproblemet är att studera interaktionen mellan student och artificiell intelligens. Vi delar in undersökningen av interaktionen i två delar. För det första att undersöka i vilken utsträckning som studenter har tillit till förutsägelserna av studieresultat från en artificiell intelligens. För det andra att observera studenters reaktion på förutsägelserna under pågående utbildning. 

Undersökningarnas kontext avgränsas till matematiktunga kurser på universitetsnivå. En orsak till avgränsningen är att inom naturvetenskap och teknik pågår generellt kurser under åtta veckor, vilket medför att undersökningar kan itereras inom tidsintervall på cirka två månader. En ytterligare orsak är att matematiktunga kurser möjliggör att samla in data för studenters studiemönster. Möjligheten uppstår på grund av att dessa kurser generellt innehåller beräkningsuppgifter vars svar är kvantitativa.

Analysen av behov och krav samt utvecklingen av funktionalitet till webbsidan avgränsar vi till studenters perspektiv. Därav analyseras inte lärares behov och krav, och det utvecklas ingen funktionalitet för lärare under arbetets gång. Avgränsningen görs dels för att minska omfånget av analysen samt för att det är studenterna som genererar data till förutsägelserna av studieresultat. I framtida studier kan det vara intressant att även undersöka lärares behov och krav för att utveckla verktyg till dem. Utvecklandet av webbsidan begränsas av att webbsidan behöver vara skalbar och flexibel. Dessutom begränsas konstruerandet av webbsidan av att utvecklingstiden för en första iteration måste vara mindre än sju veckor för att rymmas inom arbetets tidsram.

Utvecklandet av djupinlärningsalgoritmer för att förutsäga studieresultat avgränsas till tre specifika studieresultat. Dessa studieresultat är: underkänt eller godkänt betyg, graderat betyg enligt skalan U, 3, 4 och 5 samt skrivningspoäng för tentamensskrivning. Vi motiverar valet av de tre studieresultaten genom att de är kvantitativa mått med olika grad av upplösning som alla är lämpliga att använda för djupinlärning. Till sist avgränsas undersökningen till att modellera aleatoriska osäkerheter (definierade i avsnitt \ref{NN_and_uncert}) för studieresultatet skrivningspoäng. Vi väljer avgränsningen för att minska undersökningens omfång och för att vårt syfte med studien är att visa på en vidare utvecklingspotential, snarare än att utveckla en fullständig modell för förutsägelsernas osäkerheter.

\section{Rapportstruktur och innehåll}
Utvecklandet av webbsidan tillhör området informationsteknologi och webbutveckling medan analys av behov, krav och användarinteraktion tillhör området design och produktutveckling. På samma sätt tillhör utvecklandet av djupinlärningsalgoritmer forskningsfältet artificiell intelligens med djupinlärning som ett underområde. Arbetet utgör därför en tvärvetenskaplig studie som kan sättas i sammanhang med \emph{Edtech-industrin}.

De tre disciplinerna inom arbetet kopplar emellertid tätt till varandra. En representation av kopplingarna ges i figur \ref{fig:project_structure}. Exempelvis medför behov- och kravanalysen metodval för webbsidan som i sin tur genererar data till utvecklingen av djupinlärningsalgoritmerna. På liknande vis återkopplar webbsidan med användarstatistik till produktutvecklingen och djupinlärningen möjliggör för att undersöka interaktionen mellan student och artificiell intelligens.

\begin{figure}[hbtp]
    \centering
    \hspace{25px}
    \resizebox {0.5\textwidth} {!} {
        \definecolor{klight_green_400}{RGB}{156, 204, 101}

\tikzset{%
  project part/.style={
    circle,
    draw,
    fill=klight_green_400,
    thick,
    minimum size=1cm
  },
  main line/.style={
    draw,
    line width=0.25mm,
    opacity=1,
    minimum size=1cm
  },
}

\begin{tikzpicture}[x=1.5cm, y=1.5cm, ->,>=stealth',auto, thick]
% Base project nodes
\node [project part/.try] (web) at (0,0) {$\textbf{W}$};
\node [project part/.try] (pl) at (2,0) {$\textbf{P}$};
\node [project part/.try] (dl) at (1,-1.732) {$\textbf{D}$};

% Connect them 
\path[main line/.style={font=\sffamily\small}]
    (dl) edge[bend right] node [right] {Interaktion} (pl)
    (web) edge[bend right] node [left] {Data} (dl)
    (pl) edge[bend right] node [above, midway] {Behov} (web)
    (web) edge[bend right] node [below, midway] {Statistik} (pl);
\end{tikzpicture}
    }
    \caption{Den tvärvetenskapliga kopplingen mellan de i detta arbete samarbetande områdena webbutveckling (W), produktutveckling (P) och djupinlärning (D). }
    \label{fig:project_structure}
\end{figure}

Kopplingarna avspeglas även i kommande avsnitts struktur och innehåll. Rapporten är strukturerad kapitelvis utifrån de tre delproblemen: utveckling av webbsida, utveckling av artificiell intelligens samt interaktion mellan student och artificiell intelligens. I kapitel \ref{sec: webb} presenteras analysen av studenternas behov och krav samt utvecklandet av webbsidan. I kapitel \ref{sec:Deep} presenteras undersökningen av att utveckla en artificiell intelligens för att förutsäga studieresultat. Därefter appliceras utvecklad artificiell intelligens genom webbsidan i kapitel \ref{sec: inter} för att undersöka interaktionen mellan student och artificiell intelligens. Till sist presenteras en övergripande diskussion samt resonemang kring fortsatta studier med utgångspunkt från det här arbetet.







%Från fackspråk: ''de tre delarna i arbetet ska komma som en naturlig följd för läsaren utifrån syftena och problemen´´. Med andra ord, de tre delarna är en konsekvens av problemen vi försöker lösa och INTE tvärt om. Här ska vi även betona styrkan i det tvärventenskapliga.

% \vspace{+3cm}

% \textbf{gammalt:}
% lämnar det också spår i form av kvantitativa data som kan vittna om studenternas studiemönster. Eftersom denna data är personligt genererad är det möjligt att den innehåller information om framtida studieprestationer på individuell nivå. Utan extra organisation skulle elever kunna få information om deras sannolika slutresultat. Nödvändiga åtgärder skulle därför kunna tas i tid för att exempelvis undvika ett underkänt resultat. På samma sätt kan lärare få insikter, på en ny nivå, om hur eleverna ligger till och därefter anpassa undervisningen. På ett högre plan kan även institutionen, vilken är ytterst ansvarig för att studenterna tar sig igenom utsatt utbildning, finna värde i denna typ an information.



% \vspace{+20px}
% (tidigare inledning)

% Undervisning och bedömning är två av utbildningssystemets huvuduppgifter. Det generella tillvägagångssättet består klassiskt av först en undervisningsperiod som följs utav en bedömning av elevens kunskapsnivå genom exempelvis ett test. Slutligen kvantifieras normalt bedömning i ett betyg som återkopplas till eleven. Ett problem som därför kan uppstå är att varken eleven eller läraren har kännedom om elevens kunskapsnivå förrän efter bedömning. En bedömning som sker i slutskedet då det är för sent för att genomföra förändringar och förbättringar.

% En möjlig lösning är att genomföra flera prov under undervisningsperioden, men detta kräver tid och energi ifrån både elever och lärare som istället kan fokuseras på inlärningen. På grund av dessa begränsningar har det därför utvecklats alternativa metoder för att försöka förutsäga elevers studieprestationer. Ett exempel är appliceringen av maskininlärning för att förutsäga om elever som skriver in sig på så kallade \emph{Massive Open Online Courses} åstadkommer ett godkänt resultat. Speciellt har metoder baserade på djupinlärning visat en lovande potential.

% Djupinlärning kräver att det finns tillgänglig data över elevers studiemönster som korrelerar till motsvarande studieprestationer. Inom djupinlärning är det vanligt att använda internet för att samla in information. Orsaken är att webbsidor kan på ett resurseffektivt sätt samla in stora datamängder utan att kräva ansträngning och tid av användaren. För utbildning finns det flera specialutvecklade webbsidor. Tre exempel är \emph{MapleTA} \cite{MapleTA}, \emph{Piazza} \cite{Piazza} och av Göteborgs Universitet utvecklade \emph{OpenTA} \cite{OpenTA}. Av dessa tre är det enbart \emph{OpenTA} som samlar in data som är tillgänglig för det här arbetet. Datamängden ifrån \emph{OpenTA} är dock ej detaljrik och relativt liten jämfört med andra konventionella datamängder inom djupinlärning.

% Därtill tillkommer problemet att även om det finns en webbsida som infångar data följer inte att en elev självmant använder sig av webbsidan. Om eleverna inte använder webbsidan finns det ingen data varav försöken med djupinlärning misslyckas. Utifrån elevens perspektiv krävs det att webbsidan medför ett specifikt värde vilket genererar en motivation till användning. Det är därför nödvändigt att analysera elevens behov och krav för att optimera det tillföra värdet och därmed användningen.

% Ytterligare utmaningar uppstår i kontakten med eleverna. Det är okänt hur eleverna reagerar på en artificiell intelligens med förmåga att förutsäga deras studieresultat inom en lärandekontext. Exempelvis kan en negativ förutsägelse, som ett icke godkänt resultat, medföra nedbrytande konsekvenser för mottagarens motivation. Likväl kan förutsägelsen påverka positivt genom att eleven tidigt i inlärningsfasen får information om att en förändring av dess studiemönster borde genomföras. För att ovan beskriven förutsägelse därmed ska medföra ett värde för eleven krävs det att interaktionen mellan elev och artificiell intelligens undersöks.

% På förhand existerar ingen plattform som medför en uppställning för att undersöka ovan problem. Det finns därmed ett behov av att utveckla en webbsida som möjliggör en mer detaljerad datainsamling och har förmågan att koppla samman elever med artificiell intelligens för att undersöka interaktionen. Dessutom behöver webbsidan utvecklas med idéen att generera ett värde för eleven för att optimera användningen.


